% ------------------------------------------------------------------------------------------------
%	PACKAGE LIST
% ------------------------------------------------------------------------------------------------
\usepackage{
booktabs,
fontspec,
fontenc,
graphicx,
multicol,
pgfplots,
ragged2e,
tabularx,
tikz,
wasysym,
hyperref,
hanging,
multirow,
eso-pic,
comment
}

\usepackage{fontspec}
\usepackage{fontawesome}
\newfontfamily{\FA}{FontAwesome}
%\usepackage[table,x11names]{xcolor}
%\usepackage{hyperref}
%\hypersetup{colorlinks, urlcolor = magenta}
\usepackage{color, colortbl}
\definecolor{SpringGreen}{rgb}{0, 1, 0.498}
\definecolor{brightpink}{rgb}{1.0, 0.0, 0.5}

%\graphicspath{{../figs/}}

\mode<presentation>

% ------------------------------------------------------------------------------------------------
%	TABLE OF CONTENTS
% ------------------------------------------------------------------------------------------------
\useoutertheme[subsection=false,shadow]{miniframes}
\setbeamertemplate{section in toc}[sections numbered]
\setbeamertemplate{subsection in toc}[subsections numbered]

% ------------------------------------------------------------------------------------------------
%	ITEMIZE
% ------------------------------------------------------------------------------------------------
\setbeamertemplate{itemize item}{$\bullet$}
\setbeamertemplate{itemize subitem}{$\circ$ }
\setbeamertemplate{itemize subsubitem}{$\bullet$}

\setlength{\parskip}{0.5em}

% ------------------------------------------------------------------------------------------------
%	COLORS
% ------------------------------------------------------------------------------------------------

% sthlm Colors
\definecolor{sthlmLightBlue}{RGB}{90,200,250}
\definecolor{sthlmBlue}{RGB}{52,170,220}
\definecolor{sthlmDarkBlue}{RGB}{0,122,255}
\definecolor{sthlmLightRed}{RGB}{255,45,85}
\definecolor{sthlmRed}{RGB}{255,59,48}
\definecolor{sthlmLightYellow}{RGB}{255,204,0}
\definecolor{sthlmYellow}{RGB}{255,149,0}
\definecolor{sthlmPurple}{RGB}{88,86,214}
\definecolor{sthlmGreen}{RGB}{76,217,100}
\definecolor{sthlmGrey}{RGB}{142,142,147}
\definecolor{sthlmLightGrey}{RGB}{233,233,233}
\definecolor{sthlmDarkGrey}{RGB}{61,61,70}
\definecolor{brewblue}{RGB}{19, 36, 72}
\definecolor{brewgreen}{RGB}{59, 134, 134}
\definecolor{brewsage}{RGB}{121, 189, 154}
\definecolor{brewfoam}{RGB}{168, 219, 168}
\definecolor{brewlight}{RGB}{207, 240, 158}

% General
\setbeamercolor{normal text}{fg=sthlmDarkGrey}
\hypersetup{colorlinks=true, urlcolor=sthlmDarkBlue, linkcolor=sthlmDarkBlue}
\setbeamercolor{structure}{fg=sthlmDarkGrey}
\setbeamercolor{alerted text}{fg=sthlmRed}
\setbeamercolor{example text}{fg=white}
\setbeamercolor{copyright text}{fg=sthlmLightBlue}
\setbeamercolor{palette primary}{fg=sthlmDarkGrey}
\setbeamercolor{palette secondary}{fg=sthlmDarkGrey,bg=sthlmLightGrey}
\setbeamercolor{palette tertiary}{fg=black,bg=sthlmDarkGrey}
\setbeamercolor{palette quaternary}{fg=white, bg=sthlmDarkGrey}

\setbeamercolor{mini frame}{bg=sthlmLightGrey}
\setbeamercolor{section in head/foot}{fg=sthlmDarkGrey, bg=sthlmLightGrey}

% Titlepage
\setbeamercolor{title}{parent=normal text}
\setbeamercolor{subtitle}{parent=normal text}
\setbeamercolor{institute}{parent=normal text}

% Content
\setbeamercolor{frametitle}{parent=palette quaternary}

% Blocks
\setbeamercolor{block title}{fg=white,bg=sthlmDarkBlue}
\setbeamercolor{block body}{fg=sthlmDarkGrey, bg=sthlmLightGrey}
\setbeamercolor{block title example}{fg=black, bg=sthlmGreen}
\setbeamercolor{block body example}{fg=sthlmDarkGrey, bg=sthlmLightGrey}
\setbeamercolor{block title alerted}{fg=black, bg=sthlmLightRed}
\setbeamercolor{block body alerted}{fg=sthlmDarkGrey, bg=sthlmLightGrey}

% Notes
\setbeamercolor{note page}{fg=sthlmDarkGrey,bg=sthlmLightGrey}
\setbeamercolor{note title}{fg=white, bg=sthlmDarkGrey}
\setbeamercolor{note date}{parent=note title}

% Page Number
\setbeamercolor{page number in head/foot}{fg=sthlmDarkGrey}

\setbeamercolor{qed}{fg=sthlmGrey}
\setbeamercolor{itemize item}{fg=sthlmDarkBlue}
\setbeamercolor{itemize subitem}{fg=sthlmDarkBlue}
\setbeamercolor{itemize subsubitem}{fg=sthlmDarkBlue}

\renewcommand\UrlFont{\texttt}

% ------------------------------------------------------------------------------------------------
%	FONTS
% ------------------------------------------------------------------------------------------------

% General

% Declare fontfamilys
\if@doNoFlama%
	% Sans serif math option
	\if@doSans%
	% Sans serif math
		\usepackage{fontspec}%
		\setmainfont{Helvetica Neue Light}%
	\else%
		% Serif math
		\usefonttheme{professionalfonts}%
		\usepackage[no-math]{fontspec}%
	\fi%
	
	\newfontfamily\Light{Helvetica Neue UltraLight}%
	\newfontfamily\Book{Helvetica Neue Bold}%
	\newfontfamily\bfseries{Helvetica Neue Medium}%
	\setsansfont{Helvetica Neue Light}%
\else%
	% Sans serif math option
	\if@doSans%
	% Sans serif math
		\usepackage{fontspec}%
		\setmainfont{Roboto Light}%
	\else%
		% Serif math
		\usefonttheme{professionalfonts}%
		\usepackage[no-math]{fontspec}%
	\fi%
	
	%\newfontfamily\Light{Roboto Light}
	%\newfontfamily\Medium{Roboto Medium}
	%\newfontfamily\Regular{Roboto Regular}
	%\newfontfamily\Italic{Roboto Italic}
	%\newfontfamily\LightItalic{Roboto Light Italic}
	%\newfontfamily\MediumItalic{Roboto Medium Italic}
	%\newfontfamily\Black{Roboto Black}
	%\newfontfamily\Bold{Roboto Bold}
	%\newfontfamily\BlackItalic{Roboto Black Italic}
	%\newfontfamily\BoldItalic{Roboto Bold Italic}
	\newfontfamily\Thin{Roboto Thin}
	\newfontfamily\ThinItalic{Roboto Thin Italic}
	%\newfontfamily\bfseries{Roboto Bold}%
	%\setsansfont{Roboto Thin}%
	%\newfontfamily\texttt{SourceCodePro-Light}
\fi%

%\setmainfont{Roboto Light}
%\newfontfamily\Light{Roboto Light}
%\newfontfamily\Medium{Roboto Medium}
%\newfontfamily\Regular{Roboto Regular}
%\newfontfamily\Italic{Roboto Italic}
%\newfontfamily\LightItalic{Roboto Light Italic}
%\newfontfamily\MediumItalic{Roboto Medium Italic}
%\newfontfamily\Black{Roboto Black}
%\newfontfamily\Bold{Roboto Bold}
%\newfontfamily\BlackItalic{Roboto Black Italic}
%\newfontfamily\BoldItalic{Roboto Bold Italic}
%\newfontfamily\Thin{Roboto Thin}
%\newfontfamily\ThinItalic{Roboto Thin Italic}

% Font sizes

% Titlepage
\setbeamerfont{title}{family=\bfseries,size=\fontsize{24}{26}}
\setbeamerfont{subtitle}{family=\ThinItalic,size=\fontsize{14}{18}}
%\setbeamerfont{subtitle}{shape=\ThinItalic}
%\setbeamerfont{subtitle}{family=\ThinItalic,size={\fontsize{14}{18}}}
%\setbeamerfont{subtitle}{family=\fontfamily{Thin Italic}\selectfont}
%\setbeamerfont{subtitle}{\fontfamily{Thin Italic}},size=\fontsize{14}{18}}
\setbeamerfont{date}{size=\fontsize{10}{14}}
\setbeamerfont{author}{family=\bfseries,size=\fontsize{13}{15}}
\setbeamerfont{institute}{size=\fontsize{09}{10}}

% Section
\setbeamerfont{section title}{size*={39pt}{24pt}, family = \bfseries, series=\bfseries}% Content
\setbeamerfont{frametitle}{family=\bfseries,size=\large}
%\setbeamerfont{copyright text}{family=\Light,size=\tiny}
%\setbeamerfont{block title}{family=\Book,size=\large}
%\setbeamerfont{block title alerted}{family=\Book,size=\large}
\setbeamerfont{alerted text}{family=\bfseries}

% Captions
%\setbeamerfont{caption name}{family=\Book}

% ------------------------------------------------------------------------------------------------
%	TITLE PAGE
% ------------------------------------------------------------------------------------------------

% Titlepage structure
\def\maketitle{\ifbeamer@inframe\titlepage\else\frame[plain]{\titlepage}\fi}
\def\titlepage{\usebeamertemplate{title page}}
\setbeamertemplate{title page}
%\frame[plain]{\titlepage}
{
	% Add background to title page
  	%\AddToShipoutPictureFG*{\includegraphics[width=\paperwidth]{backgroundiegs.pdf}}
	\AddToShipoutPictureFG*{\includegraphics[width=\paperwidth]{background_armor.pdf}}
	\begin{minipage}[b][\paperheight]{\textwidth}
	%\vspace*{5mm}
	%\includegraphics[height=14mm]{./logo}\par
	\vspace*{20mm}
	\ifx\insertsubtitle\@empty%
	\else%
		{\usebeamerfont{title}\usebeamercolor[fg]{title}\MakeUppercase{\inserttitle}\par}%
	\fi%
	\ifx\insertsubtitle\@empty%
	\else%
		{\usebeamerfont{subtitle}\usebeamercolor[fg]{subtitle}\insertsubtitle\par}%
		\vspace*{5mm}
	\fi%
	\ifx\insertdate\@empty%
	\else%
		{\usebeamerfont{date}\usebeamercolor[fg]{date}\insertdate\par}%
	\fi% 
	
	\vfill
	
	\ifx\insertauthor\@empty%
	\else%
		{\usebeamerfont{author}\usebeamercolor[fg]{author}\insertauthor\par}%
	\fi%
	\ifx\insertinstitut\@empty%
	\else%
		%\vspace*{3mm}
		{\usebeamerfont{institute}\usebeamercolor[sthlmRed]{institute}\insertinstitute\par}%
	\fi% 
	\vspace*{7.5mm}
	\end{minipage}
}

% ------------------------------------------------------------------------------------------------
%	SECTION PAGES
% ------------------------------------------------------------------------------------------------

% Make Sectionhead uppercase
\newcommand{\insertsectionHEAD}{%
	\expandafter\insertsectionHEADaux\insertsectionhead}
	\newcommand{\insertsectionHEADaux}[3]{\MakeUppercase{#3}
}

\if@doSectionPage\@empty
\else
% Insert frame with section title at every section start
\AtBeginSection[]
{
\begingroup
\setbeamercolor{background canvas}{bg=sthlmDarkGrey}
\begin{frame}[plain]
\centering
\vfill\usebeamerfont{section title}\textcolor{white}{\insertsectionHEAD}\vfill
\end{frame}
\endgroup
}
\fi

% ------------------------------------------------------------------------------------------------
%	HEADLINE
% ------------------------------------------------------------------------------------------------
\makeatletter
\def\progressbar@progressbar{} % the progress bar
\newcount\progressbar@tmpcounta% auxiliary counter
\newcount\progressbar@tmpcountb% auxiliary counter
\newdimen\progressbar@pbht %progressbar height
\newdimen\progressbar@pbwd %progressbar width
\newdimen\progressbar@tmpdim % auxiliary dimension

\progressbar@pbwd=\paperwidth
\progressbar@pbht=1.0ex

% the progress bar
\def\progressbar@progressbar{%
    \progressbar@tmpcounta=\insertframenumber
    \progressbar@tmpcountb=\inserttotalframenumber
    \progressbar@tmpdim=\progressbar@pbwd
        \divide\progressbar@tmpdim by 1000
    \multiply\progressbar@tmpdim by \progressbar@tmpcounta
    \divide\progressbar@tmpdim by \progressbar@tmpcountb
    	\multiply\progressbar@tmpdim by 1000
  \begin{tikzpicture}[very thin]

    \shade[top color=sthlmLightGrey,bottom color=sthlmLightGrey,middle color=sthlmLightGrey]
      (0pt, 0pt) rectangle ++ (\progressbar@pbwd, \progressbar@pbht);

      \shade[draw=sthlmDarkBlue,top color=sthlmDarkBlue,bottom color=sthlmDarkBlue,middle color=sthlmDarkBlue] %
        (0pt, 0pt) rectangle ++ (\progressbar@tmpdim, \progressbar@pbht);

  \end{tikzpicture}%
}

\setbeamertemplate{headline}{

  \begin{beamercolorbox}[wd=\paperwidth,ht=1.0ex,center,dp=1ex]{sthlmLightGrey}%
    \progressbar@progressbar%
  \end{beamercolorbox}%
}

% ------------------------------------------------------------------------------------------------
%	FRAME TITLE
% ------------------------------------------------------------------------------------------------
\setbeamertemplate{frametitle}
{
\begin{beamercolorbox}[wd=\paperwidth,leftskip=0.3cm,rightskip=0.3cm,ht=3ex,dp=1.5ex]{frametitle}
	 \usebeamerfont{frametitle}\MakeUppercase{\insertframetitle}%
\end{beamercolorbox}
}

% ------------------------------------------------------------------------------------------------
%	FOOTLINE
% ------------------------------------------------------------------------------------------------
\usenavigationsymbolstemplate{}
\setbeamertemplate{footline}
{
  \leavevmode%
  \hbox{%
  \begin{beamercolorbox}[wd=.333333\paperwidth,ht=2.25ex,dp=1ex, center]{author in head/foot}%
    \usebeamerfont{author in head/foot}\insertshortauthor
  \end{beamercolorbox}%
  \begin{beamercolorbox}[wd=.333333\paperwidth,ht=2.25ex,dp=1ex, center]{date in head/foot}%
    \usebeamerfont{date in head/foot}\insertshortdate
  \end{beamercolorbox}%
  \begin{beamercolorbox}[wd=.333333\paperwidth,ht=2.25ex,dp=1ex, center]{date in head/foot}%
    \usebeamerfont{number in head/foot}\insertframenumber{} / \inserttotalframenumber\hspace*{2ex} 
  \end{beamercolorbox}}%
  \vskip0pt%
}

% ------------------------------------------------------------------------------------------------
%	CAPTIONS
% ------------------------------------------------------------------------------------------------
\setbeamertemplate{caption label separator}{: }

% ------------------------------------------------------------------------------------------------
%	BLOCKS
% ------------------------------------------------------------------------------------------------
\setbeamertemplate{block begin}
{
  \setbeamercolor{item}{parent=block body}
  \par\vskip\medskipamount%
  \begin{beamercolorbox}[sep=.5ex,dp=0.6ex,leftskip=0.5ex,rightskip=0.5ex]{block title}
    \usebeamerfont*{block title}\insertblocktitle%
  \end{beamercolorbox}%
  {\parskip0pt\par}%
  {\nointerlineskip\vskip-0.5pt}%
  \usebeamerfont{block body}%
  \begin{beamercolorbox}[sep=.5ex,dp=0.6ex,leftskip=0.5ex,rightskip=0.5ex,vmode]{block body}%
}
\setbeamertemplate{block end}  
{\end{beamercolorbox}\vskip\smallskipamount}

\setbeamertemplate{block alerted begin}
{
  \setbeamercolor{item}{parent=block body alerted}
  \par\vskip\medskipamount%
  \begin{beamercolorbox}[sep=.5ex,dp=0.6ex,leftskip=0.5ex,rightskip=0.5ex]{block title alerted}
    \usebeamerfont*{block title alerted}\insertblocktitle%
  \end{beamercolorbox}%
  {\parskip0pt\par}%
  {\nointerlineskip\vskip-0.5pt}%
  \usebeamerfont{block body alerted}%
  \begin{beamercolorbox}[sep=.5ex,dp=0.6ex,leftskip=0.5ex,rightskip=0.5ex,vmode]{block body alerted}%
}
\setbeamertemplate{block alerted end}
{\end{beamercolorbox}\vskip\smallskipamount}

\setbeamertemplate{block example begin}
{
  \par\vskip\medskipamount%
  \begin{beamercolorbox}[sep=.5ex,dp=0.6ex,leftskip=0.5ex,rightskip=0.5ex]{block title example}
    \usebeamerfont*{block title example}\insertblocktitle%
  \end{beamercolorbox}%
  {\parskip0pt\par}%
  {\nointerlineskip\vskip-0.5pt}%
  \usebeamerfont{block body example}%
  \begin{beamercolorbox}[sep=.5ex,dp=0.6ex,leftskip=0.5ex,rightskip=0.5ex,vmode]{block body example}%
}
\setbeamertemplate{block example end}
{\end{beamercolorbox}\vskip\smallskipamount}

% ------------------------------------------------------------------------------------------------
%	BLOCK HOVERING ABOVE THE SLIDE
% ------------------------------------------------------------------------------------------------

\newcommand<>{\hover}[1]{\uncover#2{%
	\begin{tikzpicture}[remember picture,overlay]%
	\draw[fill,opacity=0.4] (current page.south west)
	rectangle (current page.north east);
	\node at (current page.center) {#1};
	\end{tikzpicture}}
	}

% ------------------------------------------------------------------------------------------------
%	VERTICALLY ALIGNED COLUMNS
% ------------------------------------------------------------------------------------------------
	
\usepackage{environ}% Required for \NewEnviron, i.e. to read the whole body of the environment

\newcounter{acolumn}%  Number of current column
\newlength{\acolumnmaxheight}%   Maximum column height


% `column` replacement to measure height
\newenvironment{@acolumn}[1]{%
    \stepcounter{acolumn}%
    \begin{lrbox}{\@tempboxa}%
    \begin{minipage}{#1}%
}{%
    \end{minipage}
    \end{lrbox}
    \@tempdimc=\dimexpr\ht\@tempboxa+\dp\@tempboxa\relax
    % Save height of this column:
    \expandafter\xdef\csname acolumn@height@\roman{acolumn}\endcsname{\the\@tempdimc}%
    % Save maximum height
    \ifdim\@tempdimc>\acolumnmaxheight
        \global\acolumnmaxheight=\@tempdimc
    \fi
}

% `column` wrapper which sets the height beforehand
\newenvironment{@@acolumn}[1]{%
    \stepcounter{acolumn}%
    % The \autoheight macro contains a \vspace macro with the maximum height minus the natural column height
    \edef\autoheight{\noexpand\vspace*{\dimexpr\acolumnmaxheight-\csname acolumn@height@\roman{acolumn}\endcsname\relax}}%
    % Call original `column`:
    \orig@column{#1}%
}{%
    \endorig@column
}

% Save orignal `column` environment away
\let\orig@column\column
\let\endorig@column\endcolumn

% `columns` variant with automatic height adjustment
\NewEnviron{acolumns}[1][]{%
    % Init vars:
    \setcounter{acolumn}{0}%
    \setlength{\acolumnmaxheight}{0pt}%
    \def\autoheight{\vspace*{0pt}}%
    % Set `column` environment to special measuring environment
    \let\column\@acolumn
    \let\endcolumn\end@acolumn
    \BODY% measure heights
    % Reset counter for second processing round
    \setcounter{acolumn}{0}%
    % Set `column` environment to wrapper
    \let\column\@@acolumn
    \let\endcolumn\end@@acolumn
    % Finally process columns now for real
    \begin{columns}[#1]%
        \BODY
    \end{columns}%
}

% ------------------------------------------------------------------------------------------------
%	IMAGES
% ------------------------------------------------------------------------------------------------

\newbox\mytempbox
\newdimen\mytempdimen

\newcommand\includegraphicscopyright[3][]{%
  \leavevmode\vbox{\vskip3pt\raggedright\setbox\mytempbox=\hbox{\includegraphics[#1]{#2}}%
    \mytempdimen=\wd\mytempbox\box\mytempbox\par\vskip1pt%
    \usebeamerfont{copyright text}{\usebeamercolor[fg]{copyright text}{\vbox{\hsize=\mytempdimen#3}}}\vskip3pt%
}}

\mode
<all>